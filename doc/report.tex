\documentclass[a4paper,10pt]{article} % 改为 10pt 以节省空间

% --- 包的引入 ---
\usepackage{ctex} % 中文支持
\usepackage{geometry} % 页面边距设置
\usepackage{graphicx} % 插入图片
\usepackage{amsmath} % 数学公式
\usepackage{amssymb} % 数学符号
\usepackage{listings} % 代码块
\usepackage{xcolor} % 颜色
\usepackage{hyperref} % 超链接
\usepackage{tikz} % 绘图
\usetikzlibrary{shapes.geometric, arrows, positioning, fit, calc}
\usepackage{array} % 表格增强
\usepackage{booktabs} % 三线表
\usepackage{enumitem} % 列表环境
\usepackage{titlesec} % 控制章节间距(用于紧凑布局)

% --- 页面布局 --- 
\geometry{a4paper, top=1.8cm, bottom=1.8cm, left=1.8cm, right=1.8cm} % 缩小页边距

% --- 紧凑排版设置 ---
% \linespread{0.95}                % 行距略减小
% \setlength{\parskip}{0pt}       % 段间距为 0
\setlength{\parindent}{1em}     % 首行缩进
\setlist[itemize]{noitemsep, topsep=1pt, parsep=0pt, partopsep=0pt}     % itemize 更紧凑
\setlist[enumerate]{noitemsep, topsep=1pt, parsep=0pt, partopsep=0pt}   % enumerate 更紧凑
\titlespacing*{\section}{0pt}{6pt}{4pt}
\titlespacing*{\subsection}{0pt}{6pt}{3pt}
\titlespacing*{\subsubsection}{0pt}{4pt}{2pt}

% --- TikZ 样式定义(缩小节点与文字) ---
\tikzstyle{process} = [rectangle, rounded corners, minimum width=1.6cm, minimum height=0.6cm, text centered, draw=black, fill=blue!10, font=\small]
\tikzstyle{io} = [trapezium, trapezium left angle=70, trapezium right angle=110, minimum width=1.4cm, minimum height=0.6cm, text centered, draw=black, fill=orange!20, font=\small]
\tikzstyle{decision} = [diamond, aspect=1.3, minimum width=1.4cm, text centered, draw=black, fill=green!15, font=\small]
\tikzstyle{arrow} = [very thin, ->, >=stealth]
\tikzstyle{connector} = [very thin, <->, >=stealth]
\tikzstyle{line} = [very thin, -, >=stealth]
\tikzstyle{container} = [rectangle, rounded corners, draw=gray, dashed, inner sep=0.15cm]

\title{\textbf{Lab 1:FTP 客户端与服务器实现报告}}
\author{姓名:李孙木铎 \quad 学号:2023010304 \quad 班级:软件31}
\begin{document}

\maketitle

\section{项目概述}
本项目旨在通过 Socket 编程,从零开始构建一个简易的 FTP (File Transfer Protocol) 客户端和服务器应用。项目分为两部分:首先,通过一个简单的 UDP 编程任务熟悉 Socket API;然后,重点实现一个支持多客户端并发、具备核心 FTP 功能(如用户认证、文件传输、目录操作等)的 C 语言服务器,以及一个功能匹配的 Python 客户端。

\subsection{FTP 协议简介}
FTP 是一种用于在网络上进行文件传输的应用层协议。其独特之处在于使用两个独立的 TCP 连接来完成任务:
\begin{itemize}
    \item \textbf{控制连接 (Control Connection):} 用于传输客户端命令和服务器响应,在整个会话期间保持活动。
    \item \textbf{数据连接 (Data Connection):} 用于传输实际的文件内容或目录列表,是临时性的,每次传输前建立,传输后关闭。
\end{itemize}
数据连接的建立方式分为主动模式 (PORT) 和被动模式 (PASV),本项目均予以支持。

\section{UDP 编程实践}
此部分作为 Socket 编程的入门实践,要求修改已有的 UDP 客户端和服务器代码。
\begin{itemize}
    \item \textbf{服务器修改:} 服务器在接收到客户端消息时,为其附加一个从1开始递增的序列号,然后将“序列号 + 原始消息”返回给客户端。
    \item \textbf{客户端修改:} 客户端循环发送数字0到50给服务器,并接收服务器返回的带序列号的消息。
\end{itemize}

通过此实践,我们熟悉了 UDP Socket 的基本用法,如 \verb|socket()|, \verb|bind()|, \verb|sendto()|, \verb|recvfrom()| 等函数的使用。此外,我们还理解了 UDP 的无连接特性及其不可靠性,这为后续 FTP 服务器的实现打下了基础。

对于“如何用 UDP 实现聊天程序”的问题,核心思路是引入一个中心服务器作为消息中继。客户端将消息(包含目标用户标识)发送给服务器,服务器根据目标标识将消息转发给对应的客户端。为解决 UDP 的不可靠性,可在应用层引入序列号和确认机制(ACK)来保证消息的有序和可靠到达。

\section{FTP 服务器与客户端设计}

\subsection{系统整体架构}
系统由服务器和客户端两部分组成,它们通过网络进行交互。服务器在指定端口监听控制连接请求,并为数据传输动态创建数据连接。

\begin{figure}[h!]
    \centering
    \begin{tikzpicture}[scale=0.707, every node/.style={transform shape}, node distance=2.5cm and 3cm, auto]
        % 节点
        \node [process] (client) {FTP 客户端 (Python)};
        \node [process, right=of client] (server) {FTP 服务器 (C)};
        \node [io, below=0.8cm of client] (client_data) {数据传输};
        \node [io, below=0.8cm of server] (server_data) {数据传输};

        % 连线
        \draw [connector] (client) -- node[midway, above] {控制连接 (TCP)} node[midway, below] {命令/响应} (server);
        \draw [connector, dashed] (client_data) -- node[midway, above] {数据连接 (TCP)} (server_data);
        
        % 关系
        \draw [arrow, gray] (client.south) -- ++(0, -0.4) -| (client_data.north);
        \draw [arrow, gray] (server.south) -- ++(0, -0.4) -| (server_data.north);
    \end{tikzpicture}
    \caption{FTP 系统高层架构图}
    \label{fig:arch}
\end{figure}

\subsection{服务器端设计 (C语言)}
服务器采用多进程并发模型,主进程负责监听连接,每当有新客户端连接时,通过 `fork()` 创建一个子进程专门服务于该客户端,从而实现多客户端的并发处理。

\subsubsection{核心工作流程}
每个子进程都遵循一个清晰的状态机来处理客户端会话,如图所示。
\begin{figure}[h!]
    \centering
    % 缩放为线性尺寸的 0.5,并确保节点随缩放变形
    \begin{tikzpicture}[scale=0.5, every node/.style={transform shape}, node distance=1.13cm and 0.71cm, auto]
        % 节点
        \node [process] (start) {发送 220 欢迎消息};
        \node [decision, below=of start] (auth) {等待认证};
        \node [process, below=of auth] (cmd_loop) {已登录,处理命令循环};
        \node [decision, below=of cmd_loop] (is_quit) {收到 QUIT?};
        \node [process, right=of cmd_loop] (handle_cmd) {执行具体命令 (RETR, STOR, CWD...)};
        \node [process, below=of is_quit] (end) {发送 221 告别,关闭连接};

        % 连线
        \draw [arrow] (start) -- (auth);
        \draw [arrow] (auth) -- node[right] {认证成功} (cmd_loop);
        \draw [arrow] (cmd_loop) -- (is_quit);
        \draw [arrow] (is_quit) -- node[right] {是} (end);
        \draw [arrow] (is_quit.west) -- ++(-1, 0) |- node[left, pos=0.25] {否} (cmd_loop.west);
        \draw [arrow] (cmd_loop.east) -- (handle_cmd.west);
        \draw [arrow] (handle_cmd.south) .. controls +(down:1cm) and +(right:1cm) .. (cmd_loop.south);
    \end{tikzpicture}
    \caption{服务器子进程工作流程}
    \label{fig:server_flow}
\end{figure}

\subsubsection{模块化设计}
为保证代码的清晰和可维护性,服务器代码被拆分为多个模块:
\begin{itemize}
    \item \textbf{main.c:} 主程序入口,负责参数解析、Socket 初始化、监听和 `fork()` 子进程。
    \item \textbf{handle.c:} 连接处理模块,负责单个客户端会话的完整生命周期,是命令分发的核心。
    \item \textbf{connect.c:} 连接管理模块,专门处理 `PORT` 和 `PASV` 命令,并建立数据连接。
    \item \textbf{file.c:} 文件与目录操作模块,实现 `RETR`, `STOR`, `CWD`, `PWD`, `MKD`, `RMD`, `LIST` 等命令的逻辑。
    \item \textbf{utils.c:} 工具模块,提供发送响应、读取命令行等通用功能。
\end{itemize}

\subsection{客户端设计 (Python)}
客户端提供两个实现:基于命令行的 client.py 与基于 PyQt5 的 gui.py。下图给出客户端的逻辑框架与关键模块。

\begin{figure}[h!]
    \centering
    \begin{tikzpicture}[scale=0.6, node distance=1.2cm and 2.8cm, auto, every node/.style={transform shape}]
        % 节点
        \node[process, align=center] (ui) {用户 \\(stdin / GUI)};

        \node[process, right=of ui, align=center] (client) {client.py \\ (命令解析 \& 控制连接)};

        \node[process, right=of client, align=center] (server) {FTP 服务器 \\ (控制/数据两通道)};
        \node[io, below=of client, align=center] (datalogic) {数据逻辑 \\ PASV/PORT 处理 \\ RETR / STOR / LIST};
        \node[io, below=of ui, align=center] (gui) {gui.py \\ (QProcess 启动 client.py)};

        % 连线
        \draw[arrow] (ui) -- node[midway, above] {命令 / 行输入} (client);
        \draw[arrow] (client) -- node[midway, above] {控制连接 (TCP)} (server);
        \draw[arrow] (client.south) -- (datalogic.north);
        \draw[arrow] (gui) -- node[midway, left] {stdin/stdout} (client.west);

        % 数据连接示意(虚线)
        \draw[connector, dashed] (datalogic.east) .. controls +(right:8mm) and +(down:5mm) .. node[midway, right] {data socket (PASV/PORT)} (server.south);
    \end{tikzpicture}
    \caption{客户端逻辑框架图(简洁视图)}
    \label{fig:client_flow}
\end{figure}

\subsection{已实现命令列表}
本项目实现了 `guide.md` 中要求的所有核心 FTP 命令,如下表所示。

\begin{table}[h!]
    \centering
    \begin{tabular}{lll}
        \toprule
        \textbf{类别} & \textbf{命令} & \textbf{功能描述} \\
        \midrule
        认证 & \verb|USER|, \verb|PASS| & 匿名用户登录认证 \\
        连接模式 & \verb|PORT|, \verb|PASV| & 主动模式与被动模式的数据连接建立 \\
        文件传输 & \verb|RETR|, \verb|STOR| & 文件下载与上传 \\
        目录操作 & \verb|CWD|, \verb|PWD|, \verb|MKD|, \verb|RMD|, \verb|LIST| & 切换目录、显示当前目录、创建/删除目录、列出目录内容 \\
        系统控制 & \verb|SYST|, \verb|TYPE|, \verb|QUIT| & 获取服务器系统类型、设置传输类型、退出会话 \\
        \bottomrule
    \end{tabular}
    \caption{已实现的 FTP 命令}
\end{table}


\subsection{命令实现要点(代码层面)}
每个命令在代码中的处理方式如下:

\begin{itemize}
    \item \textbf{USER / PASS}(\verb|handle.c|)\\
    会话状态机处理登录:客户端发送 \verb|USER| / \verb|PASS|,服务器在 \verb|handle_connection| 中返回相应状态码并在通过后将会话置为已登录。

    \item \textbf{PORT}(客户端:\verb|client.py|;服务器:\verb|connect.c|)\\
    客户端建立本地 listener,并发送 \verb|PORT h1,...,p2|;服务器解析参数保存为 \verb|session->data_addr|,模式设为主动(PORT)。

    \item \textbf{PASV}(客户端:\verb|client.py|;服务器:\verb|connect.c|)\\
    服务器创建监听 socket(端口0)、返回 \verb|227| 含地址与端口;客户端解析并在传输时 connect 到该地址。

    \item \textbf{数据连接建立}(\verb|connect.c: establish_data_connection|)\\
    根据 \verb|session->mode| 执行 \verb|connect()|(PORT)或 \verb|accept()|(PASV),上层据此发送 150/425/426 等响应。

    \item \textbf{RETR}(客户端:\verb|client.py|;服务器:\verb|file.c|)\\
    客户端发送 \verb|RETR|,建立数据通道并接收写入本地文件;服务器验证路径(\verb|is_path_safe|)、打开文件、通过数据通道发送文件并返回结果码。

    \item \textbf{STOR}(客户端:\verb|client.py|;服务器:\verb|file.c|)\\
    客户端发送 \verb|STOR| 并通过数据通道上传本地文件;服务器验证目标路径、打开目标文件并写入接收的数据,完成后返回结果码。

    \item \textbf{LIST}(客户端:\verb|client.py|;服务器:\verb|file.c|)\\
    客户端请求 \verb|LIST|,建立数据通道并显示收到的目录文本;服务器生成目录列表并通过数据通道发送。

    \item \textbf{CWD / PWD / MKD / RMD}(\verb|file.c|)\\
    客户端发送控制命令并显示响应;服务器对路径做规范化与安全检查,分别调用 \verb|chdir|/\verb|getcwd|/\verb|mkdir|/\verb|rmdir| 并返回状态码。

    \item \textbf{SYST / TYPE / QUIT}(\verb|handle.c|)\\
    简单控制命令:\verb|SYST| 返回系统类型,\verb|TYPE| 处理传输类型,\verb|QUIT| 返回会话统计并关闭会话。

\end{itemize}

\subsection{挑战与解决方案}
\begin{enumerate}[label=\arabic*)]
    \item \textbf{并发模型选择:}
    \begin{itemize}
        \item \textbf{挑战:} FTP 服务器必须能同时服务多个客户端。
        \item \textbf{方案:} 采用了多进程模型 (\verb|fork()|),实现简单,且进程间天然隔离,稳定性高,非常适合本项目的需求。
    \end{itemize}
    
    \item \textbf{数据连接管理:}
    \begin{itemize}
        \item \textbf{挑战:} 正确处理 \verb|PORT| 和 \verb|PASV| 两种数据连接模式。
        \item \textbf{方案:} 在会话结构体 \verb|connection| 中引入 \verb|data_conn_mode_t| 枚举来跟踪当前模式。
        \item 
        \verb|establish_data_connection()| 函数根据此状态来决定是 \verb|connect()| 到客户端(PORT模式)还是 \verb|accept()| 来自客户端的连接(PASV模式)。
    \end{itemize}

    \item \textbf{路径安全与规范化:}
    \begin{itemize}
        \item \textbf{挑战:} 客户端可能发送包含 \verb|.|、\verb|..| 或绝对路径的请求,必须防止其访问 FTP 根目录之外的文件系统。
        \item \textbf{方案:} 设计了 \verb|is_path_safe()| 函数。它首先将用户提供的路径与当前工作目录拼接成一个绝对路径,然后通过字符串处理将路径规范化(例如,将 \verb|/a/b/../c| 转换为 \verb|/a/c|),最后用 `strncmp` 检查规范化后的路径是否以服务器设定的根目录为前缀。
    \end{itemize}
\end{enumerate}

\subsection{附加功能}
\begin{itemize}
    \item \textbf{简易 GUI:} 实现了基于 PyQt5 的图形化客户端,提升了用户体验。通过 \verb|QProcess| 将后端逻辑与前端界面解耦,结构清晰。
    \item \textbf{传输统计:} 在 \verb|QUIT| 命令的响应中,增加了本次会话总传输字节数的统计信息,为用户提供了有用的反馈。
\end{itemize}

\end{document}